\documentclass[OMG]{article}

\usepackage[utf8]{inputenc}
%\usepackage[cp1250]{inputenc}
\usepackage[IL2]{fontenc}
\usepackage[czech]{babel}
\usepackage{soul}
\usepackage{soulutf8}
\usepackage{enumerate}
\usepackage{booktabs}
\usepackage{amsmath}
\begin{document}

\begin{equation}
f(x) = \frac{A_0}{2} + \sum \limits_{n=1}^{\infty} A_n \cos \left( \frac{2 n \pi x}{\nu} - \alpha_n \right) 
\end{equation}
\begin{tabular}{  l  l  c  }
\midrule[\heavyrulewidth]
\multicolumn{3}{c}{\textbf{Results}} \\
\midrule[\heavyrulewidth]
cat & dog icecream & eath, ipad  \\ \hline
\textbf{sattelite-1} & 4 october 1957 & 83.6 \\ 

\multicolumn{3}{c}{whats next ěščř} \\

sattelite-2 & 3 november 1957 & 508.3  \\ 
sattelite-1 & 1 febuary 1958 & 21.5 \\
\midrule[\heavyrulewidth]
\end{tabular}
\begin{equation*}
|x| = 
\begin{cases}
	1 & x \in [ 0;\infty ] \\
	0 &	x = 0 \\
	-1 & x \in [-\infty; 0]
\end{cases}
\end{equation*}



\begin{equation}
X_n = X_k \quad \mbox{právě když} \quad Y_n = Y_k \text{ a } Z_n = Z_k
\end{equation}
%\noindent \textbf{Zápočtový test z předmětu 01PSL — akademický rok 2017/2018}
%Zápočtový test z předmětu 01PSL — akademický rok 2017/2018 \\
\section*{Zápočtový $x_{i,j}$ test z $A_1 \in B_1 $ předmětu$N\times M$ 01PSL — akademický rok 2017/2018}
\textbf{Roman Yaremchuk}

\begin{equation}
    y_1=u+v, \quad y_2=-\frac{u+v}{2}+\frac{\mathrm{i}}{2}\sqrt{3}(u-v),
    \quad y_3=-\frac{u+v}{2}-\frac{\mathrm{i}}{2}\sqrt{3}(u-v),
\end{equation}

\begin{equation}
u=\sqrt[3]{-q+\sqrt{q^2+p^3}},\qquad v=\sqrt[3]{-q-\sqrt{q^2+p^3}}
\end{equation}
\begin{equation}
\{ F \}_{n=1}^\infty
\end{equation}
\[
    y_1=u+v, \quad y_2=-\frac{u+v}{2}+\frac{\mathrm{i}}{2}\sqrt{3}(u-v),
    \quad y_3=-\frac{u+v}{2}-\frac{\mathrm{i}}{2}\sqrt{3}(u-v),
  \]
  kde
  \[
    u=\sqrt[3]{-q+\sqrt{q^2+p^3}},\qquad v=\sqrt[3]{-q-\sqrt{q^2+p^3}}.
    \mathcal{A + B}
  \]
\textbf{\section*{Text}}
\noindent Běžný $180^\circ$ (tak zvaný \textit{odstavcový}) text je pro zpracování \LaTeX em možné připravit v jakémkoli
textovém $\Bigg)$.editoru. Je potřeba mít $\alpha_0$ na paměti několik pravidel [1]:

\begin{equation}
f(x,y, \alpha , \beta) = \iint \frac{\sum \limits_{n=1}^{\infty} x^{n + 1} }{\cos x + \sin x}
\end{equation}

\begin{enumerate}
\item Mezery
\begin{enumerate}[(i)]
\item libovolný počet po sobě jdoucích mezer ve vstupním textu se chová při zpracování
jako jedna mezera,
\item mezeru na začátku rádku \LaTeX \ \ \ ignoruje
\end{enumerate} 
\item Nový odstavec se vyznačuje (alespoń) jdním prázdným řádkem.
\end{enumerate}

\textbf{\section{Matematika}}





\end{document}